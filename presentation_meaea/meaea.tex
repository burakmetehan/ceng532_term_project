%\documentclass[10pt,notes]{beamer}       % print frame + notes
%\documentclass[10pt, notes=only]{beamer}   % only notes
\documentclass[11pt]{beamer}              % only frames

%%%%%% IF YOU WOULD LIKE TO CREATE LECTURE NOTES COMMENT OUT THE FOlLOWING TWO LINES
%\usepackage{pgfpages}
%\setbeameroption{show notes on second screen=bottom} % Both

\usepackage{graphicx}
\DeclareGraphicsExtensions{.pdf,.png,.jpg}
\usepackage{color}
\usetheme{winslab}
\usepackage[utf8]{inputenc}
\usepackage[english]{babel}
\usepackage{amsmath}
\usepackage{amsfonts}
\usepackage{amssymb}




\usepackage{algorithm2e,algorithmicx,algpseudocode}
\algnewcommand\Input{\item[\textbf{Input:}]}%
\algnewcommand\Output{\item[\textbf{Output:}]}%
\newcommand\tab[1][1cm]{\hspace*{#1}}

\algnewcommand{\Implement}[2]{\item[\textbf{Implements:}] #1 \textbf{Instance}: #2}%
\algnewcommand{\Use}[2]{\item[\textbf{Uses:}] #1 \textbf{Instance}: #2}%
\algnewcommand{\Trigger}[1]{\Statex{\textbf{Trigger:} (#1)}}%
\algnewcommand{\Events}[1]{\item[\textbf{Events:}] #1}%
\algnewcommand{\Need}[1]{\item[\textbf{Needs:}] #1}%
\algnewcommand{\Event}[2]{\Statex \item[\textbf{On#1:}](#2) \textbf{do}}%
\algnewcommand{\Trig}[3]{\State \textbf{Trigger}  #1.#2 (#3) }%
\def\true{\textbf{T}}
\def\false{\textbf{F}}


\author[Burak Metehan Tunçel]{Burak Metehan Tunçel\\\href{mailto:tuncel.burak@metu.edu.tr}{tuncel.burak@metu.edu.tr}}
%\author[J.\,Doe \& J.\,Doe]
%{%
%  \texorpdfstring{
%    \begin{columns}%[onlytextwidth]
%      \column{.45\linewidth}
%      \centering
%      John Doe\\
%      \href{mailto:john@example.com}{john@example.com}
%      \column{.45\linewidth}
%      \centering
%      Jane Doe\\
%      \href{mailto:jane.doe@example.com}{jane.doe@example.com}
%    \end{columns}
%  }
%  {John Doe \& Jane Doe}
%}

\title[WINS Beamer Template]{Mutual Exclusion in Distributed Systems}
\subtitle[Short SubTitle]{Balancing Efficiency and Safety in Distributed Systems}
%\date{} 

\begin{document}

\begin{frame}[plain]
\titlepage
\note{In this talk, I will present .... Please answer the following questions:
\begin{enumerate}
\item Why are you giving presentation?
\item What is your desired outcome?
\item What does the audience already know  about your topic?
\item What are their interests?
\item What are key points?
\end{enumerate}
}
\end{frame}

\begin{frame}[label=toc]
    \frametitle{Outline of the Presentation}
    \tableofcontents[subsubsectionstyle=hide]
\note{ The possible outline of a talk can be as follows.
\begin{enumerate}
\item Outline 
\item Problem and background
\item Design and methods
\item Major findings
\item Conclusion and recommendations 
\end{enumerate} Please select meaningful section headings that represent the content rather than generic terms such as ``the problem''. Employ top-down structure: from general to more specific.
}
\end{frame}
%
%\part{This the First Part of the Presentation}
%\begin{frame}
%        \partpage
%\end{frame}
%
\section{The Problem}
%\begin{frame}
%        \sectionpage
%\end{frame}

\begin{frame}{The problem}
\framesubtitle{Tell a \alert{STORY} from the background to the conclusion}

\begin{block}{The Problem Name} 
\begin{itemize}
  \item Ensuring data consistency and preventing race conditions is crucial. When multiple processes attempt to access and modify shared resources simultaneously, some unpredictable outcomes and data corruption can be experienced.
  \item Mutual Exclusion (ME) algorithms dictate the order in which processes interact with critical sections.
  \item By ensuring only one process executes within a critical section at a time, ME algorithms uphold data integrity.
  \item There is need for a ME algorithm that is deadlock-free, message-efficient.
\end{itemize}
\end{block}

\note{}
\end{frame}

\section{The Contribution}
\begin{frame}
\frametitle{What is the solution/contribution}
\framesubtitle{}

\begin{itemize}
  \item The realm of Mutual Exclusion (ME) algorithms in distributed systems boasts a rich history.
  \item Each iteration building upon the strengths and addressing the limitations of its predecessor.
  \item The Agrawal-El Abbadi algorithm builds upon these existing solutions, aiming to achieve a balance between message complexity and deadlock freedom.
  \item Introduces the concept of quorums, subsets of processes that must grant permission for a process to enter the critical section.
  \item Reduces message overhead compared to algorithms requiring communication with all processes.
\end{itemize}

\end{frame}


\section{Motivation/Importance}
\begin{frame}
\frametitle{Importance of the Problem}
\framesubtitle{}

\begin{itemize}
  \item \textbf{Data Integrity:} Ensures only one process can modify a shared resource at a time, maintaining data accuracy.
  \item \textbf{Resource Allocation:} Manages limited resources efficiently among processes, avoiding starvation.
  \item \textbf{Avoiding Deadlocks:} Prevents or resolves deadlock situations, where processes wait indefinitely for resources.
  \item \textbf{System Reliability:} Contributes to stable system operations by managing access to shared resources.
\end{itemize}

\end{frame}

\begin{frame}
\frametitle{Motivation Behind the Agrawal-El Abbadi Algorithm}
\framesubtitle{}

\begin{itemize}
  \item \textbf{Efficiency:} Aims to reduce communication overhead, making the system more efficient.
  \item \textbf{Scalability:} Designed to handle increasing numbers of processes and resources effectively.
  \item \textbf{Deadlock Prevention:} Focuses on eliminating or easily resolving deadlocks, improving system robustness.
  \item \textbf{Practicality:} Provides a straightforward and effective approach for real-world application
\end{itemize}

\end{frame}


\section{Background/Model/Definitions/Previous Works}


\subsection{Model, Definitions}

\frame{
\frametitle{Some Terminology}

\begin{itemize}
  \item \textbf{Mutual Exclusion (ME):} Prevents simultaneous access to a shared resource by multiple processes to avoid data inconsistency.
  \item \textbf{Critical Section:} A part of the system where access must be exclusive to prevent race conditions.
  \item \textbf{Message Complexity:} The number of messages needed to perform a task in a distributed system, with a goal to minimize for efficiency.
  \item \textbf{Deadlocks:} A standstill where processes wait indefinitely for each other to release resources.
  \item \textbf{Timestamps:} Used to order events or messages in a distributed system, ensuring a consistent sequence.
  \item \textbf{Quorums:} A selected group of processes whose approval is required for a process to access the critical section, reducing the need for full consensus.
  \item \textbf{Broadcasts:} Sending a message to all processes in the system, used for consistency but can increase message overhead.
\end{itemize}
}

\subsection{Background, Previous Works}
\begin{frame}{Background}

The journey towards efficient mutual exclusion (ME) in distributed systems has seen significant milestones, each addressing the challenges of maintaining data consistency, reducing message complexity, and preventing deadlocks.

\begin{itemize}
  \item Lamport's Bakery Algorithm (1978):
    \begin{itemize}
      \item \textbf{Overview:} Introduced a ticket-based system where processes obtain numbers to ensure orderly access to the critical section.
      \item \textbf{Pros:} Simple and easy to implement.
      \item \textbf{Cons:} High message overhead due to system-wide broadcasts.
    \end{itemize}
  
    \item Ricart-Agrawala Algorithm (1981):
      \begin{itemize}
        \item \textbf{Overview:} Utilizes timestamps in request messages, granting access based on the precedence of timestamps.
        \item \textbf{Pros:} Reduces message complexity compared to Lamport's algorithm.
        \item \textbf{Cons:} Risk of deadlocks due to potential circular waiting.
      \end{itemize}
\end{itemize}
\end{frame}

\subsection{Background, Previous Works}
\begin{frame}{Background}

  \begin{itemize}
    \item Maekawa's Voting Algorithm (1985):
      \begin{itemize}
        \item \textbf{Overview:} Implements a voting mechanism among processes, allowing entry to the critical section through a virtual election.
        \item \textbf{Pros:} Eliminates deadlocks.
        \item \textbf{Cons:} Can result in high message overhead in larger systems.
      \end{itemize}
    
    \item Towards the Agrawal-El Abbadi Algorithm:
      \begin{itemize}
        \item Building on these foundational works, the Agrawal-El Abbadi algorithm introduces quorums -specific subsets of processes- to grant access to the critical section, aiming for an optimal balance between message efficiency and deadlock prevention.
      \end{itemize}
  \end{itemize}
\end{frame}



\section{Contribution}
\subsection{Main Point 1}
\begin{frame}{Main Point 1: A Figure}
\framesubtitle{Abstract the Major Results}

TODO: Complete after code implementation and experiments.

\end{frame}


\section{Experimental results/Proofs}

\subsection{Main Result 1}
\begin{frame}
\frametitle{Main Result 1}
\framesubtitle{}

TODO: Complete after code implementation and experiments.

\end{frame}


\section{Conclusions}
\begin{frame}
\frametitle{Conclusions}
\framesubtitle{}

TODO: Complete after code implementation and experiments.

\end{frame}

\section*{References}
\begin{frame}{References}
\tiny
\bibliographystyle{IEEEtran}
\bibliography{refs}
\end{frame}



\thankslide




\end{document}