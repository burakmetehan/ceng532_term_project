%\documentclass[10pt,notes]{beamer}       % print frame + notes
%\documentclass[10pt, notes=only]{beamer}   % only notes
\documentclass[11pt]{beamer}              % only frames

%%%%%% IF YOU WOULD LIKE TO CREATE LECTURE NOTES COMMENT OUT THE FOlLOWING TWO LINES
%\usepackage{pgfpages}
%\setbeameroption{show notes on second screen=bottom} % Both

\usepackage{graphicx}
\DeclareGraphicsExtensions{.pdf,.png,.jpg}
\usepackage{color}
\usetheme{winslab}
\usepackage[utf8]{inputenc}
\usepackage[english]{babel}
\usepackage{amsmath}
\usepackage{amsfonts}
\usepackage{amssymb}




\usepackage{algorithm2e,algorithmicx,algpseudocode}
\algnewcommand\Input{\item[\textbf{Input:}]}%
\algnewcommand\Output{\item[\textbf{Output:}]}%
\newcommand\tab[1][1cm]{\hspace*{#1}}

\algnewcommand{\Implement}[2]{\item[\textbf{Implements:}] #1 \textbf{Instance}: #2}%
\algnewcommand{\Use}[2]{\item[\textbf{Uses:}] #1 \textbf{Instance}: #2}%
\algnewcommand{\Trigger}[1]{\Statex{\textbf{Trigger:} (#1)}}%
\algnewcommand{\Events}[1]{\item[\textbf{Events:}] #1}%
\algnewcommand{\Need}[1]{\item[\textbf{Needs:}] #1}%
\algnewcommand{\Event}[2]{\Statex \item[\textbf{On#1:}](#2) \textbf{do}}%
\algnewcommand{\Trig}[3]{\State \textbf{Trigger}  #1.#2 (#3) }%
\def\true{\textbf{T}}
\def\false{\textbf{F}}


\author[Burak Metehan Tunçel]{Burak Metehan Tunçel\\\href{mailto:tuncel.burak@metu.edu.tr}{tuncel.burak@metu.edu.tr}}
%\author[J.\,Doe \& J.\,Doe]
%{%
%  \texorpdfstring{
%    \begin{columns}%[onlytextwidth]
%      \column{.45\linewidth}
%      \centering
%      John Doe\\
%      \href{mailto:john@example.com}{john@example.com}
%      \column{.45\linewidth}
%      \centering
%      Jane Doe\\
%      \href{mailto:jane.doe@example.com}{jane.doe@example.com}
%    \end{columns}
%  }
%  {John Doe \& Jane Doe}
%}

\title[WINS Beamer Template]{Byzantine Failures}
\subtitle[Short SubTitle]{Trust and Resilience in Distributed Systems}
%\date{} 

\begin{document}

\begin{frame}[plain]
\titlepage
\note{In this talk, I will present .... Please answer the following questions:
\begin{enumerate}
\item Why are you giving presentation?
\item What is your desired outcome?
\item What does the audience already know  about your topic?
\item What are their interests?
\item What are key points?
\end{enumerate}
}
\end{frame}

\begin{frame}[label=toc]
    \frametitle{Outline of the Presentation}
    \tableofcontents[subsubsectionstyle=hide]
\note{ The possible outline of a talk can be as follows.
\begin{enumerate}
\item Outline 
\item Problem and background
\item Design and methods
\item Major findings
\item Conclusion and recommendations 
\end{enumerate} Please select meaningful section headings that represent the content rather than generic terms such as ``the problem''. Employ top-down structure: from general to more specific.
}
\end{frame}
%
%\part{This the First Part of the Presentation}
%\begin{frame}
%        \partpage
%\end{frame}
%
\section{The Problem}
%\begin{frame}
%        \sectionpage
%\end{frame}

\begin{frame}{Clock Synchronization}
\framesubtitle{}
\begin{block}{Clock Synchronization in Distributed Systems} 

Ensuring accurate and consistent timekeeping across geographically distributed systems is fundamental for coordinated operation. Traditional clock synchronization algorithms assume processes behave correctly and exchange reliable information; however, real-world systems are not immune to failures. In some scenarios, processes may exhibit malicious or arbitrary behavior.

Therefore, developing robust clock synchronization algorithms that can tolerate these failures is crucial for ensuring the reliability and integrity of distributed systems.
\end{block}

\note{}
\end{frame}

\section{The Contribution}
\begin{frame}
\frametitle{What is the solution/contribution}
\framesubtitle{}
\begin{itemize}
  \item Robust clock synchronization algorithms tolerating Byzantine failures, which exhibit malicious or arbitrary behavior, is crucial.
  \item \textbf{The Mahaney-Schneider synchronizer} stands out as a practical and efficient solution for achieving this goal.
  \item Leverageing message exchange and statistical analysis to filter out misleading information.
  \item Converging towards a reliable estimate of the system time, even in the presence of Byzantine processes.
\end{itemize}
\end{frame}


\section{Motivation/Importance}
\begin{frame}
\frametitle{Motivation/Importance}
\framesubtitle{}

\begin{itemize}
  \item \textbf{Trust in Decentralized Systems:} BFT is essential for establishing trust in decentralized systems like blockchain, ensuring operations continue even when some participants are unreliable.
  \item \textbf{Real-World Applications:} It's crucial for safety in critical applications such as aviation and autonomous vehicles, where failure can have dire consequences.
  \item \textbf{Philosophical Reflection:} BFT prompts reflection on trust and consensus in systems where misinformation or malice is possible, highlighting its broader implications.
  \item \textbf{Evolving Distributed Systems:} As these systems grow in scale and importance, the role of BFT in ensuring resilience against problematic conditions becomes increasingly vital.
\end{itemize}
\end{frame}

\section{Background/Model/Definitions/Previous Works}


\subsection{Model, Definitions}

\frame{
\frametitle{Definitions}
\framesubtitle{}
\begin{itemize}
  \item Byzantine failures are the scenarios where processes exhibit arbitrary behavior including sending incorrect information, withholding data, or even actively trying to disrupt the synchronization process.
  \item Byzantine Fault Tolerance (BFT) is a property of distributed systems that allows them to continue operating correctly even when some of the nodes fail in arbitrary or malicious ways.
\end{itemize}
}

\subsection{Background, Previous Works}
\begin{frame}{Background}
\begin{itemize}
  \item Clock synchronization in distributed systems ensures consistent timekeeping across geographically.
  \item Traditional algorithms assume well-behaved processes such as Lamport's logical clocks and Cristian's algorithm.
  \item Traditional algorithms fail with Byzantine failures.
\end{itemize}
\end{frame}

\subsection{Background, Previous Works}
\begin{frame}{Background}
\begin{itemize}
  \item BFT protocols can tolerate Byzantine failures but are computationally complex and expensive.
  \item The Mahaney-Schneider synchronizer is a BFT clock synchronization algorithm that is efficient and robust.
  \item Work even in the presence of Byzantine failures.
\end{itemize}
\end{frame}


\section{Contribution}
\subsection{Main Point 1}
\begin{frame}{Main Point 1}
\framesubtitle{}

TODO: Complete after code implementation and experiments.

\note{
}
\end{frame}

\section{Experimental results/Proofs}

\subsection{Main Result 1}
\begin{frame}
\frametitle{Main Result 1}
\framesubtitle{}
TODO: Complete after code implementation and experiments.
\end{frame}


\section{Conclusions}
\begin{frame}
\frametitle{Conclusions}

TODO: Complete after code implementation.

\end{frame}

\section*{References}
\begin{frame}{References}
\tiny
\bibliographystyle{IEEEtran}
\bibliography{refs}
\end{frame}

\begin{frame}{How to prepare the talk?}
Please read \url{http://larc.unt.edu/ian/pubs/speaker.pdf}
\begin{itemize}
\item The Introduction:  Define the Problem,    Motivate the Audience,    Introduce Terminology,    Discuss Earlier Work,    Emphasize the Contributions of your Paper,    Provide a Road-map.
\item The Body:    Abstract the Major Results, Explain the Significance of the Results, Sketch a Proof of the Crucial Results
\item Technicalities: Present a Key Lemma, Present it Carefully
\item The Conclusion: Hindsight is Clearer than Foresight, Give Open Problems, Indicate that your Talk is Over
\end{itemize}

\note{
\begin{itemize}
\item The Introduction:  Define the Problem,    Motivate the Audience,    Introduce Terminology,    Discuss Earlier Work,    Emphasize the Contributions of your Paper,    Provide a Road-map.
\item The Body:    Abstract the Major Results, Explain the Significance of the Results, Sketch a Proof of the Crucial Results
\item Technicalities: Present a Key Lemma, Present it Carefully
\item The Conclusion: Hindsight is Clearer than Foresight, Give Open Problems, Indicate that your Talk is Over 
\end{itemize}
}
\end{frame}



\thankslide




\end{document}